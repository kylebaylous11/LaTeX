\documentclass[12pt]{article}


%THIS IS AMS 333 HW 4 FILE


\usepackage{graphicx}
%\usepackage{float}
\usepackage{caption}
\usepackage[font=footnotesize]{caption}
%\usepackage{article}
\usepackage{amsmath}
\usepackage{amssymb}
\usepackage{amsthm}
\usepackage[dvips]{epsfig}

\usepackage[left=1in,right=1in,bottom=1.5in,top=1.0in]{geometry}
\setlength{\topmargin}{0.1in}

\makeatletter
\def\@seccntformat#1{%
  \expandafter\ifx\csname c@#1\endcsname\c@section\else
  \csname the#1\endcsname\quad
  \fi}
\makeatother

\begin{document} 

\begin{titlepage}
   \begin{center}
   
       \vspace*{1cm}
 
       \textbf{\Large{AMS 333: Homework 4}}
 
       \vspace{0.5cm}
 
       \vspace{1.5cm}
 
       \textbf{\Large{Kyle Baylous}}\\
       \vspace{1.5cm}
       \textbf{\Large{SBU ID: 111374388}}
 
       \vfill

 
       \vspace{0.8cm}
 
      % \includegraphics[width=0.4\textwidth]
 
       Applied Mathematics and Statistics\\
       Stony Brook University\\
       11/26/19
 
   \end{center}
\end{titlepage}

\section{Introduction to Epidemiology and The SIR Model}

Analysis of the spread of infectious disease is a crucial component of mathematical biology and of great importance to public health. In particular, one of the most fundamental questions in public health is whether or not we can fully understand how an infectious disease propagates throughout a population [1]. There exist many mathematical models that aim to describe a variety of infectious diseases and each one considers a variety of problems and specific parameters. For example, the effectiveness of vaccination or quarantine can be analyzed in many models in order to see how they prevent the occurrence of epidemics [1]. Epidemics are abnormally large numbers of infection cases occurring within a short time period. Within these models, there usually exists three types of populations: The susceptible population, the infected population and the recovered population [1]. However, the dynamics of the model as a whole can be analyzed mainly through studying the infected population dynamics. In particular, the SIR model, which is a model that describes disease with an acquired immunity, is widely used to study the dynamics of epidemics like Influenza. The SIR model can be described as shown below, and we can utilize this model to analyze the dynamics of the H1N1 strain of Influenza using infection case data over the course of about 30 weeks. The variables S, I and R represent the susceptible, infected and recovered host densities, respectively. The parameters $\beta$ and $\gamma$ represent the transmission probability and recovery rate (with immunity), respectively. It is important to note that N = S + I + R. 

\begin{gather*}
\dfrac{dS}{dt} = -{\beta}S\dfrac{I}{N}
\end{gather*}
\begin{gather*}
\dfrac{dI}{dt} = {\beta}S\dfrac{I}{N}-{\gamma}I
\end{gather*}
\begin{gather*}
\dfrac{dR}{dt} = {\gamma}I
\end{gather*}
 
\section{Influenza Epidemics: H1N1 Strain}
The passage of a given strain of H1N1 can be modeled as an SIR disease in which individuals become infected once, and then gain immunity. Data was collected for influenza in a population of 7000 individuals (initial susceptible population), and 10 cases were initially identified (initial infected population). The initial recovered population is taken to be zero. The data can be modeled as shown in Figure 1.
%1
\begin{center}
   \includegraphics[scale=0.35]{/Users/Kyle/Desktop/AMS333_HW4_PartA.jpg}
   \captionof{figure}{Plot showing the number of infected cases versus time (weeks). The peak number of infection cases is about 386. Refer to Appendix A.1 for MATLAB code.} 
\end{center}

The infectious period can be presumed to be about 1.4286 weeks, and so the value for gamma can be found as follows:

\begin{gather*}
\dfrac{1}{\gamma} = 1.4286
\end{gather*}
\begin{gather*}
{\gamma} = 0.6999 \approx0.7
\end{gather*}\

We can implement an SIR model by finding the following parameters and using a time step of 0.001 weeks to model the infected population for approximately 30 weeks. The recovery rate calculated previously, ${\gamma}$, can be used along with $R_0$, the number of secondary cases generated by a typical infectious individual during its period of infectiousness in an entirely susceptible population, to find the parameter ${\beta}$. The parameter ${\beta}$ is the transmission probability. It is important to note that $R_0$ takes a value in the range of 1.4 to 2.8 when considering H1N1, and so the proper value must be found for this model by adjusting the $R_0$ in the model and plotting the results along with the data provided for the infected population. It was found that $R_0$=1.45 and ${\beta}$=1.015 produce the most satisfactory results and yield a model that best fits the data (using ${\beta}$ = $R_0$*${\gamma}$). The plot is shown in Figure 2.
%2
\begin{center}
   \includegraphics[scale=0.35]{/Users/Kyle/Desktop/AMS333_HW4_PartB.jpg}
   \captionof{figure}{Plot showing the implementation of the SIR model. The data for the number of infected cases versus time (weeks) along with the infected population curve from the SIR model is shown. It is important to note that an $R_0$ value of 1.45 was used to produce the curve shown, and the associated value for ${\beta}$ is 1.015. Refer to Appendix A.1 for MATLAB code.} 
\end{center}

We can now analyze the dynamics of the resulting epidemic by looking at the infected population. From the model, we can see that the initial infected population is 10 and the number of infected individuals increases sharply until about week 16, where the infected population reaches its peak at approximately 386 individuals. After it reaches this peak, which took approximately 16 weeks, the infected population declines to zero. 

By the end of the epidemic, there are about 3150 individuals who are still susceptible to the disease, meaning that they avoided the disease. However, there exist about 3860 who are recovered, meaning that by the end of the epidemic, about 3860 individuals had been infected by the disease. With a 0.12\% mortality rate, about 4.6$\approx$5 people would die.

%%%%%%%%%%%%%%% increase beta

We can increase ${\beta}$ by factor of 2 so that ${\beta}$ = 2.03. With this new model, the epidemic begins around week 2, since this is when the infected population rises rapidly, and lasts about 12 weeks until week 15 approximately. The maximum number of infected individuals is now about 2022 now. This result makes sense in terms of the parameter $R_0$, since this is the initial rate of increase of a disease over a generation and helps describe whether or not a population is at risk depending on if $R_0$ is less than or greater than 1. In this case, increasing ${\beta}$ by a factor of 2 also effectively increases the value for $R_0$, since $R_0$ = $\dfrac{\beta}{\gamma}$. Since $R_0$ is much larger now, we expect a larger amount of infection cases, which is evident in the resulting plot shown below:
%3
\begin{center}
   \includegraphics[scale=0.35]{/Users/Kyle/Desktop/AMS333_HW4_PartC.jpg}
   \captionof{figure}{Plot showing the implementation of the SIR model with ${\beta}$ increased by a factor of 2. Refer to Appendix A.2 for MATLAB code.} 
\end{center}

%%%%%%%%%%%%%%% increase beta and gamma

We can now increase ${\gamma}$ by a factor of 2 in addition to our increment of ${\beta}$. We now see that the same peak value for the infected population is obtained (2022), but the infection does not last as long in the population (only about 6-7 weeks) This is because the recovery rate is increasing and so the population of infected individuals does not persist as long as in the prior model. Ultimately, increasing ${\gamma}$ decreases the mean infectious period, or the duration of infection $\dfrac{1}{\gamma}$, and so this is why we do not see the infected individuals peaking for an extended period of time. This is shown in Figure 4.
%4
\begin{center}
   \includegraphics[scale=0.35]{/Users/Kyle/Desktop/AMS333_HW4_PartD.jpg}
   \captionof{figure}{Plot showing the implementation of the SIR model with ${\gamma}$ increased by a factor of 2 as well. Refer to Appendix A.3 for MATLAB code.} 
\end{center}

%%%%%%%%%%%%%%% decrease beta only

We can now analyze the model when decreasing ${\beta}$ by a factor of 2 and keeping ${\gamma}$ unchanged. We can see that the resulting plot does not show any peaking infected population. More specifically, with ${\beta}$ decreased by a factor of 2, the susceptible population remains approximately (6974) at its original value of 7000 and also, the number of infected and recovered individuals at the end of 30 weeks are 0 and 36, respectively. This can be explained in terms of $R_0$ because, generally, when ${\beta}$ is decreased by a factor of 2, then $R_0$ is effectively decreased by a factor of 2 as well and brings $R_0$ below 1. This means that the infection would die out, as depicted in the following figure.

%5
\begin{center}
   \includegraphics[scale=0.35]{/Users/Kyle/Desktop/AMS333_HW4_PartE.jpg}
   \captionof{figure}{Plot showing the implementation of the SIR model with ${\gamma}$ unchanged but with ${\beta}$ decreased by a factor of 2. Refer to Appendix A.4 for MATLAB code.} 
\end{center}

%%%%%%%%%%%%%%% increase gamma only

We can now analyze the model when increasing ${\gamma}$ by a factor of 2 and keeping ${\beta}$ unchanged. We can see that the resulting plot shows similar dynamics in comparison to the original model, but there do exist some differences. More specifically, with ${\gamma}$ increased by a factor of 2, the susceptible population is approximately 3150 and also, the number of infected and recovered individuals at the end of 30 weeks are 0 and 3860, respectively. This can be explained in terms of the infectious period because, generally, when ${\gamma}$ is increased by a factor of 2, then $\dfrac{1}{\gamma}$ is effectively decreased. This means that the infectious period becomes smaller and explains why the time between the rapid increase and decrease of the infected population is much smaller in comparison to the original plot. This is depicted in the following figure.
%6
\begin{center}
   \includegraphics[scale=0.35]{/Users/Kyle/Desktop/AMS333_HW4_PartF.jpg}
   \captionof{figure}{Plot showing the implementation of the SIR model with ${\beta}$ unchanged but with ${\gamma}$ increased by a factor of 2. Refer to Appendix A.5 for MATLAB code.} 
\end{center}

By adjusting these models using variations of the original parameters, we can see that the resulting dynamics of the epidemic depend mainly on the value for $R_0$ and corresponding values for the infectious period, $\dfrac{1}{\gamma}$, and the probability of transmission per contact, ${\beta}$. Ultimately, $R_0$ depends on the number of contacts and probability of transmission, and the resulting value of $R_0$ tells us whether the infection persists or not. More specifically, when $R_0<1$, then the infection dies out, and when $R_0>1$, then there is an epidemic that persists in the population. Accurate estimation of the value of the reproductive number is central in the planning of control of intervention efforts. The goal of public health interventions can often be reduced to that of bringing $R_0$ below 1. 


\subsection*{Understanding the Effects of Quarantine and Vaccination}

Quarantine and vaccination are two mechanisms that aim to reduce the spread of an epidemic. Quarantine involves identifying those who are infected and removing them from the general public, which occurs at some fixed rate, until they have recovered. On the other hand, vaccination involves sending some susceptible individuals directly to the recovered, or immune, state and this depends on the number of shots per time period. 


\subsection*{Incorporating Quarantine Into an SIR Model}

Quarantine can be incorporated into an SIR model by subtracting a quarantine term from the infected population and by adding a quarantine term to the recovered population. This is done because, if thinking about a compartmental diagram, an arrow points from the infected compartment to the quarantine compartment and then another arrow points from the quarantine compartment to the recovered compartment. In other words, this new addition has a negative effect on the number of individuals in the infected population, hence the minus sign, and a positive effect on the recovered population, hence the positive sign. A schematic diagram to demonstrate this new model is shown below with the parameter q, which demonstrates the effectiveness of the quarantine response. The model itself is also shown.
%7
\begin{center}
   \includegraphics[scale=0.15]{/Users/Kyle/Desktop/AMS333_HW4_QSchematic.jpg}
   \captionof{figure}{Schematic showing the implementation of quarantine into the SIR model. Note that the parameter q demonstrates the effectiveness of the quarantine response.} 
\end{center}

\begin{gather*}
\dfrac{dS}{dt} = -{\beta}S\dfrac{I}{N}
\end{gather*}
\begin{gather*}
\dfrac{dI}{dt} = {\beta}S\dfrac{I}{N}-{\gamma}I-qI
\end{gather*}
\begin{gather*}
\dfrac{dR}{dt} = {\gamma}I+qI
\end{gather*}


This model can be used to calculate the number of people that die form the epidemic in a similar manner to how the deaths were calculated in the earlier base model. More specifically, we only need to find the number of individuals in the recovered class to calculate the number of people that die from the epidemic by using the associated mortality rate. This can be done by taking the final recovered population at the last time step in the simulation and multiplying by the mortality rate percentage divided by 100 (to convert the percentage into a decimal). We can now implement this model in MATLAB to explore how the epidemic changes with different levels of effectiveness. It is important to note that the original amount of total deaths was about 4.6$\approx$5, and in order to reduce deaths by 50\%, 90\% and 99\%, the quarantine parameter values would have to be 0.15 (total deaths $\approx$ 2.5) , 0.3 (total deaths $\approx$ 0.5) and 0.6 (total deaths $\approx$ 0.05), respectively. In reality, this would mean that the quarantine rate would have to be 15\%, 30\% and 60\%, respectively, and so reducing the amount of deaths by a drastic amount would be difficult to do since such high quarantine rates are not very feasible. However, quarantine is very effective as shown in the figure below.  
 %8
\begin{center}
   \includegraphics[scale=0.3]{/Users/Kyle/Desktop/AMS333_HW4_QEffects.jpg}
   \captionof{figure}{Plots showing the implementation of quarantine into the SIR model for q values 0.15, 0.3 and 0.6 (top to bottom). Note that the parameter q demonstrates the effectiveness of the quarantine response and increased q values result in a decreased infected population. Refer to Appendix A.6 for MATLAB code} 
\end{center}

\subsection*{Incorporating Vaccination Into an SIR Model}

Vaccination can be incorporated into an SIR model by subtracting a vaccination term from the susceptible population. This is done because, if thinking about a compartmental diagram, an arrow points from the susceptible compartment directly to the recovered compartment. In other words, this new addition has a negative effect on the number of individuals in the susceptible population, hence the minus sign, since susceptible individuals can go right to the recovered population. Note that the vaccination does not influence the recovered population here because we do not consider vaccinated individuals in the total death calculation from the epidemic, which directly involves the use of the recovered population. A schematic diagram to demonstrate this new model is shown below with the parameter p, which demonstrates the vaccination rate. The model itself is also shown.
%9
\begin{center}
   \includegraphics[scale=0.45]{/Users/Kyle/Desktop/AMS333_HW4_VSchematic.jpg}
   \captionof{figure}{Schematic showing the implementation of vaccination into the SIR model. Note that the parameter p demonstrates the vaccination rate. Diagram adapted from Dr. Kamal Barley's lecture notes.} 
\end{center}

\begin{gather*}
\dfrac{dS}{dt} = -{\beta}S\dfrac{I}{N}-pS
\end{gather*}
\begin{gather*}
\dfrac{dI}{dt} = {\beta}S\dfrac{I}{N}-{\gamma}I
\end{gather*}
\begin{gather*}
\dfrac{dR}{dt} = {\gamma}I
\end{gather*}

This model can be used to calculate the number of people that die form the epidemic in a similar manner to how the deaths were calculated in the earlier models. More specifically, we only need to find the number of individuals in the recovered class to calculate the number of people that die from the epidemic by using the associated mortality rate. This can be done by taking the final recovered population at the last time step in the simulation and multiplying by the mortality rate percentage divided by 100 (to convert the percentage into a decimal). We can now implement this model in MATLAB to explore how the epidemic changes with different levels of vaccination. It is important to note that the original amount of total deaths was about 4.6$\approx$5, and in order to reduce deaths by 50\%, 90\% and 99\%, the vaccination rate values would have to be 0.045 (total deaths $\approx$ 2.5) , 0.28 (total deaths $\approx$ 0.5) and 0.99 (total deaths $\approx$ 0.05), respectively. Reducing the amount of deaths by a drastic amount would be similar in difficulty for 50\% and 90\% in comparison to the quarantine model since 4.5\% and 28\% of the susceptible population would have to be vaccinated, respectively. However, reducing deaths by about 99\% would be extremely difficult and not feasible since 99\% of the susceptible population would have to be vaccinated. The results of this investigation are shown in the figure below. 
%10
\begin{center}
   \includegraphics[scale=0.3]{/Users/Kyle/Desktop/AMS333_HW4_VEffects.jpg}
   \captionof{figure}{Plots showing the implementation of vaccination into the SIR model for p values 0.045, 0.28 and 0.99 (top to bottom). Note that the parameter p demonstrates the vaccination rate and increased p values result in a decreased infected population. Refer to Appendix A.7 for MATLAB code} 
\end{center}

Overall, both quarantine and vaccination are effective mechanisms in reducing the infected population size as shown in the previous figures. However, increased vaccination rates are not as effective as increased quarantine rates. In other words, from this analysis, quarantine seems to be the better mechanism in reducing the spread of an epidemic and is more feasible in comparison to the vaccination mechanism. Therefore, one of the best intervention strategies that public health programs can implement is quarantine procedures.


\subsection*{Summary of Results}

In summary, the SIR model is an excellent model to help investigate the spread of disease. It was found that $R_0$=1.45 and ${\beta}$=1.015 produce the most satisfactory results and yield a model that best fits the data for the provided H1N1 data. From the original model, we can see that the initial infected population is 10 and the number of infected individuals increases sharply until about week 16, where the infected population reaches its peak at approximately 386 individuals. After it reaches this peak, which took approximately 16 weeks, the infected population declines to zero. By the end of the epidemic, there are about 3150 individuals who are still susceptible to the disease, meaning that they avoided the disease. However, there exist about 3860 who are recovered, meaning that by the end of the epidemic, about 3860 individuals had been infected by the disease. With a 0.12 percent mortality rate, about 4.6$\approx$5 people would die. When increasing and decreasing ${\beta}$, we see that the infected population sharply increases and drastically decreases, respectively, due to how the nature of the transmission probability effects the value of $R_0$ and the resulting dynamics. When increasing and decreasing ${\gamma}$, we see that the time of infection decreases and increases, respectively, due to how the recovery rate effects the value of $R_0$, the infectious period, and the resulting dynamics. From adjusting these models using variations of the original parameters, we can see that the resulting dynamics of the epidemic depend mainly on the value for $R_0$ and corresponding values for the infectious period, $\dfrac{1}{\gamma}$, and the probability of transmission per contact, ${\beta}$. Ultimately, $R_0$ depends on the number of contacts and probability of transmission, and the resulting value of $R_0$ tells us whether the infection persists or not. More specifically, when $R_0<1$, then the infection dies out and when $R_0>1$, then there is an epidemic that persists in the population. Accurate estimations of the value of the reproductive number are central in the planning of control of intervention efforts. The goal of public health interventions can often be reduced to that of bringing $R_0$ below 1. Mechanisms for reducing the spread of disease include quarantine and vaccination. Overall, both quarantine and vaccination are effective mechanisms in reducing the infected population size. However, increased vaccination rates are not as effective as increased quarantine rates. In other words, from this analysis, quarantine seems to be the better mechanism in reducing the spread of an epidemic and is more feasible in comparison to the vaccination mechanism and the associated rates. Therefore, the best intervention strategy, from this analysis, that public health programs can implement in order to damage $R_0$ and get $R_0<1$ is quarantine procedures.

 
 \vspace{1cm}

%\bibliographystyle{siam}
%\bibliography{refstmp}

\begin{flushleft}
References:\\
\bibliography{refs}

[1] AMS 333: Mathematical Biology, Lecture Notes. "Chapter 4: Mathematical Epidemiology".

%[2] Barley, Kamal. "Lecture 3.1.1-3.1.2". Lecture, Stony Brook University, October 3, 2019.

%Include MATLAB Appendix
\end{flushleft}
\end{document}

